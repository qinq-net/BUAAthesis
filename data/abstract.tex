% !Mode:: "TeX:UTF-8"

% 中英文摘要
\begin{cabstract}
本篇文章主要介绍一套用于HIRFL-CSR外靶实验(CEE)模拟与分析的软件框架。本文首先介绍
了CEE实验的概况和主要涉及的探测器原理与布置,之后介绍了本软件框架的基本原理、主要
结构与关键技术难点的实现方法,最后给出了探测器模拟的一组结果和分析方法。

为解决对CEE实验进行模拟和未来的数据分析的问题,本工作主要在Geant4框架中构建了CEE实验
的探测器系统(低温高密核物质测量谱仪)、实现了主要探测器的数字化功能,并讨论了
径迹重建等数据分析问题。其中,本文介绍了对CEE采用的时间投影室探测器(TPC)、基于
多气隙电阻板室(MRPC)的飞行时间探测器(TOF)、多丝漂移室(MWDC)和零度角量能器
(ZDC)等探测器的几何构建,详细讨论了其中TPC、MRPC和MWDC的数字化实现,并分析了
示例性的模拟数据。

\end{cabstract}

\begin{eabstract}
A simulation and analyzation software framework for HIRFL-CSR External Target
Experiment (CEE) is built and introduced. Goals and detectors of CEE experiment are
reviewed as an introduction. Basic principles, main structure and implementations
for certain technological problems are introduced. A group of simulation results and
routes of analyzation are given as a conclusion.

To match the requirements of simulation and analyzation of future CEE experiments,
CEE detector system (low tempetrature high density nuclear matter chamber) is
constructed in Geant4, as well as implementations of digitization of major detectors.
Analyzation problems such as track reconstruction is also discussed. 

Geometry construction for CEE adopted detectors, including Time Projection Chamber
(TPC), Time of Flight (TOF) detectors based on Multi-gap Resistive Plate Chamber
(MRPC), Multi-Wire Drift Chamber (MWDC) and Zero Degree Chamber (ZDC).
Implementations of digitization of TPC, MRPC and MWDC are detailedly discussed.
An example simulation output is also analyzed.
\end{eabstract}
