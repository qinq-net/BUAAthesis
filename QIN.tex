% !Mode:: "TeX:UTF-8"
% vim: set foldmethod=marker sw=4 ts=4 noet:
\documentclass[bachelor,openany,oneside,color]{buaathesis}

% {{{ 导入宏包
\usepackage{tikz,amsmath,amssymb}% for common math symbols
	\usetikzlibrary{positioning,snakes}
%%\usepackage{gensymb}%other sumbles as \degree
\usepackage{booktabs}%for table \toprule and \bottomrule
\usepackage{physics}
\usepackage{bm}%for \bm{vector}
\usepackage{braket}%for \bra and \ket
%\usepackage{mathtools}%for correct \sum
\usepackage{commath}%for \abs
%\usepackage{mhchem} %for chemistry symbol \cm
\usepackage{mathrsfs}%for \mathscr characters
\usepackage{chngcntr,hyperref}%for hyperlink of reference
\usepackage{etoolbox}%for \ifcsdef
\usepackage{siunitx}%for units
	\DeclareSIUnit\year{a}
	\DeclareSIUnit\gauss{Gs}

%\usepackage[top=1in, bottom=1in, left=1.25in, right=1.25in]{geometry}% page size
%\usepackage[font=small,width=0.8\textwidth,labelfont=bf]{caption}%for caption font
%\usepackage{subcaption} %for \subcaption{} for {subfigure}
%\usepackage{longtable}%for longer tables
%\usepackage{multirow}%for combined rows
%\usepackage{changepage}%for displaying page number
%\usepackage{fancyhdr}
%	\pagestyle{fancy}
%	\fancyhead{}
%	\fancyhead[RO,LE]{\thepage}
%	\fancyhead[LO,RE]{\SUBTITLE}
%	\fancyfoot{}
%	\renewcommand{\headrulewidth}{0pt}
%	\renewcommand{\footrulewidth}{0pt}

%\usepackage{autobreak,breqn}

%%=== ntheorem ===
%\usepackage[amsmath,thmmarks]{ntheorem}% for {theorem}
%	\theoremstyle{plain}
%	\theoremheaderfont{\normalfont\rmfamily\CJKfamily{hei}}
%	\theorembodyfont{\normalfont\rm\CJKfamily{kai}}
%	\theoremindent0em
%	\theoremseparator{\hspace{1em}}
%	\theoremnumbering{arabic}
%	\theoremsymbol{■}          %symbol added after end of theorem
%	\newtheorem{theorem}{Theorem}[subsection]
%	\newtheorem{defn}{Definition}[subsection]

%%=== lstlisting ===
%\usepackage{listings,color}
%	\definecolor{mygreen}{rgb}{0,0.6,0}
%	\definecolor{mygray}{rgb}{0.9,0.9,0.9}
%	\definecolor{mymauve}{rgb}{0.58,0,0.82}
%	\lstset{ %
%	  backgroundcolor=\color{mygray},   % choose the background color; you must add \usepackage{color} or \usepackage{xcolor}; should come as last argument
%	%  basicstyle=\footnotesize,        % the size of the fonts that are used for the code
%	  breakatwhitespace=false,         % sets if automatic breaks should only happen at whitespace
%	  breaklines=true,                 % sets automatic line breaking
%	  captionpos=b,                    % sets the caption-position to bottom
%	  commentstyle=\color{mygreen},    % comment style
%	  deletekeywords={...},            % if you want to delete keywords from the given language
%	  escapeinside={\%*}{*)},          % if you want to add LaTeX within your code
%	  extendedchars=true,              % lets you use non-ASCII characters; for 8-bits encodings only, does not work with UTF-8
%	  frame=single,	                   % adds a frame around the code
%	  keepspaces=true,                 % keeps spaces in text, useful for keeping indentation of code (possibly needs columns=flexible)
%	  keywordstyle=\color{blue},       % keyword style
%	  language=Mathematica,                 % the language of the code
%	  morekeywords={*,...},           % if you want to add more keywords to the set
%	  numbers=left,                    % where to put the line-numbers; possible values are (none, left, right)
%	  numbersep=5pt,                   % how far the line-numbers are from the code
%	  numberstyle=\tiny\color{mygray}, % the style that is used for the line-numbers
%	  rulecolor=\color{black},         % if not set, the frame-color may be changed on line-breaks within not-black text (e.g. comments (green here))
%	  showspaces=false,                % show spaces everywhere adding particular underscores; it overrides 'showstringspaces'
%	  showstringspaces=false,          % underline spaces within strings only
%	  showtabs=false,                  % show tabs within strings adding particular underscores
%	  stepnumber=1,                    % the step between two line-numbers. If it's 1, each line will be numbered
%	  stringstyle=\color{mymauve},     % string literal style
%	  tabsize=2,	                   % sets default tabsize to 2 spaces
%	  title=\lstname                   % show the filename of files included with \lstinputlisting; also try caption instead of title
%	}

%% define math operators
\def\D{\mathrm{d}}
\def\Ln{\mathrm{ln}}
\def\CH{\mathrm{CH}}
\def\FWHM{\mathrm{FWHM}}
\def\figureinit{\centering\setlength\parindent{0pt}}
% }}}

\begin{document}

% {{{ 前缀部分
% 用户信息
% !Mode:: "TeX:UTF-8"

% 学院中英文名,中文不需要“学院”二字
% 院系英文名可从以下导航页面进入各个学院的主页查看
% http://www.buaa.edu.cn/xyykc/index.htm
\school
{物理科学与核能工程}{School of Physics and Nuclear Energy Engineering}

% 专业中英文名
\major
{核物理}{Nuclear Physics}

% 论文中英文标题
\thesistitle
{HIRFL-CSR外靶实验终端模拟和分析框架的构建}
{}
{Establishment of a terminal simulation and analyzation framework for
 HIRFL-CSR External Target Experiment (CEE)}
{}

% 作者中英文名
\thesisauthor
{秦雨浩}{QIN Yuhao}

% 导师中英文名
\teacher
{张高龙}{ZHANG Gaolong}
% 副导师中英文名
% 注:慎用‘副导师’,见北航研究生毕业论文规范
\subteacher{肖志刚}{Zhi-Gang XIAO}

% 中图分类号,可在 http://www.ztflh.com/ 查询
\category{O571.1}

% 本科生为毕设开始时间;研究生为学习开始时间
\thesisbegin{2018}{01}{15}

% 本科生为毕设结束时间;研究生为学习结束时间
\thesisend{2018}{05}{30}

% 毕设答辩时间
\defense{2018}{06}{06}

% 中文摘要关键字
\ckeyword{外靶实验,探测器构建,框架,数字化}

% 英文摘要关键字
\ekeyword{external target experiment, detector construction, framework, digitize}

\makeatletter
\hypersetup{
	pdftitle={\buaa@thesistitle},
	pdfauthor={\buaa@thesisauthor}
}
\makeatother
% !Mode:: "TeX:UTF-8"

% 班级
\class{141913}

% 学号
\studentID{14191033}

% 单位代码
\unicode{}

% 论文时间,用于首页
\thesisdate{2018}{06}


% 任务书信息
% !Mode:: "TeX:UTF-8"
% 任务书中的信息
%% 原始资料及设计要求
\assignReq
{\textbf{CSR外靶实验(CEE)}:利用中国科学院近代物理研究所的重离子加速器}
{提供的束流进行的重离子 碰撞外靶实验。现已立项低温高密度核物质测}
{量谱仪,即为本框架要求构建的探测器阵列。[1]}
{\textbf{终端模拟和分析框架}:构建一套软件框架,对探测器进行构建、模拟和}
{数字化(digitization) 并提供可视化等分析功能。[2][3]}
%% 工作内容
\assignWork
{构建框架至少应当完成以下工作:}
{1. 构建探测器,对每种探测器构建其几何模型与数字化模型;}
{2. 构建针对HIRFL-CSR外靶实验的粒子源和物理过程列表;}
{3. 基于外靶实验设计构建电磁场环境;}
{4. 调用框架实现对CEE实验的模拟、数字化与分析;}
{5. 利用有关框架实现可视化等分析功能。}
%% 参考文献
\assignRef
{[1] CEE 合作组.CEE 技术文档 1[R].2017.10}
{[2] GSI. About FairRoot[EB/OL]. https://fairroot.gsi.de/?q=about}
{[3] CERN. Geant4 User's Guide for Application Developers 10.3[M], 2016-12-06}
{}
{}
{}
{}
{}


% 页眉页脚样式
\pagestyle{mainmatter}
% 封面、任务书、声明
\maketitle
% 摘要
% !Mode:: "TeX:UTF-8"

% 中英文摘要
\begin{cabstract}
本篇文章主要介绍一套用于HIRFL-CSR外靶实验(CEE)模拟与分析的软件框架。本文首先介绍
了CEE实验的概况和主要涉及的探测器原理与布置,之后介绍了本软件框架的基本原理、主要
结构与关键技术难点的实现方法,最后给出了探测器模拟的一组结果和分析方法。
\end{cabstract}

\begin{eabstract}
A simulation and analyzation software framework for HIRFL-CSR External Target
Experiment (CEE) is built and introduced. Goals and detectors of CEE experiment are
reviewed as an introduction. Basic principles, main structure and implementations
for certain technological problems are introduced. A group of simulation results and
routes of analyzation are given as a conclusion.
\end{eabstract}

% 目录
\tableofcontents

% 正文页码样式
\mainmatter
% }}}

%% 正文
\chapter{绪论}
% {{{ 绪论

\section{课题背景}

在中高能粒子物理的实验研究中,对实验环境和探测器响应的模拟是实验准备阶段的一大核心
工作。根据这些模拟的结果,可以在实验准备阶段确定实验对不同探测器的性能提出的要求,
以评价和调整实验方案。目前,对粒子物理的模拟研究主要基于Geant 4模拟框架。该框架是一
套C++软件框架,提供了一套根据给定的物理过程列表追踪粒子及其次级粒子的工具。用户需要
使用C++语言编写或重载有关的类,以构建探测器、其数字化模块和场等其他环境因素。这种形
式意味着需要针对每个模拟问题构建一个新的模拟应用程序,并设计其数字化和数据存储与处
理的有关功能。

\section{国内外研究现状}

%%CEE
\subsection{CSR外靶实验(CEE)}

本研究所称的“HIRFL-CSR外靶实验”,指的是利用中科院近代物理研究所的重离子加速器提供的
束流进行重离子碰撞外靶实验(CSR External-target Experiment,CEE)。

该实验已经立项搭建低温高密核物质测量谱仪。该谱仪的测量手段为,利用大接收度的超导磁铁
,通过时间投影室、硅像素、高计数率飞行时间谱仪等多种国内首次采用的先进探测器,配合先
进的电子学技术,实现对中能重离子碰撞产物的全空间探测和鉴别。本研究的主要内容就是基于
CEE的设计构建软件层面的终端模拟和分析框架。

该实验将构建低温高密核物质测量谱仪,通过相对论重离子碰撞实验实现QCD相变。国际上已经
有多个类似的重离子碰撞实验,但基本运行在高能区。
\begin{itemize}
	\item 美国的相对论重离子对撞机(RHIC),每对核子对撞能量达\SI{200}{\giga\eV};
	\item 欧洲核子中心的大型强子对撞机(LHC),能量可达\SI{5.4}{\tera\eV},以寻找
		高温夸克胶子等离子体并探索其性质;
	\item 德国的反质子-离子研究装置(FAIR)上将要运行的压缩强子物质实验(CBM),
		每核子能量达\SI{45}{\giga\eV},主要目的是探索高密QCD相变和临界点。
\end{itemize}
国际上已有的重离子碰撞实验鲜有涉及低温而更高重子化学势(重子数密度)的区域,这一区
域恰好是HIRFL-CSR和将来HIAF装置将覆盖的区域。RIEFL-CSR能够提供$0.5\sim\SI{1.2}
	{\giga\eV}$的离子束流,QCD相图在这一区域有非常丰富的结构和与恒星演化密切相关的
状态方程信息。

%%GSI
\subsection{终端模拟和分析框架}

终端模拟和分析框架指的是构建一套软件框架,对探测器进行构建、模拟和数字化(digitize,
指将模拟中探测器灵敏的物理量转换为读出系统的实际输出)。核物理与高能物理领域内常用到
模拟框架Geant 4与分析框架ROOT。Geant 4需要为每个模拟问题专门编写程序,构建探测器并存
储所需的物理信息;ROOT主要用于对数据的统计分析。

以上述德国的FAIR装置为例,GSI开发了一套面向对象的模拟、重建和数据分析框架FairRoot。
该框架下可以将每个探测器构建为由数字化参数构建、几何参数构建、蒙特卡洛模拟等若干个
类组成的类库,还可以实现粒子轨迹的可视化等针对整个外靶实验的功能。FAIR上正在筹备的
CBM实验基于该框架构建了数据分析软件CbmRoot,其中构建了其中用到的各探测器及其布局,
能够对整个实验进行模拟、重建和事件显示等。

\section{课题目的}

本研究的主要目的是基于HIRFL-CSR外靶实验的设计和实际情况,构建终端模拟和分析框架。
经过调研,构建框架至少需要完成以下的工作。
\begin{enumerate}
	\item 构建探测器:对每种类型的探测器都至少需要完成以下工作:
	\begin{enumerate}
		\item 读取探测器的几何参数,构建探测器的几何形态;
		\item 读取和构建探测器的材料信息;
		\item 设计各种探测器的数字化模块,基于探测器的工作原理给出读出系统的输出;
	\end{enumerate}
		以实现对探测器主要功能的模拟。
	\item 基于HIRFL-CSR的实际情况,构建粒子源和物理列表;
	\item 基于外靶实验的设计,构建实验环境的电磁场等因素;
	\item 调用以上构建出来的框架,实现对整个实验的模拟、数字化和重建等功能;
	\item 调用分析框架提供的显示功能,实现对事件进行可视化的功能。
\end{enumerate}

\section{论文构成及研究过程}

\subsection{论文构成}

本论文将主要围绕CEE实验概况、模拟框架的主体结构、关键技术难点的实现和模拟样例与分析
四部分展开。其中,实验概况将介绍CEE实验的物理目标、计划选用的探测器类型与功能,以及
实验的探测器布局等信息,以作为本文的构建基础;主体结构将介绍本框架所划分的主要模块及
其功能;关键技术难点将介绍MRPC、TPC和MWDC三种探测器的数字化实现原理,以及径迹重建的
基本原理。

\subsection{研究过程}

本研究主要围绕Geant 4框架的学习和对探测器原理的学习逐渐展开,工作过程大致可以分为以
下的几个阶段。
\begin{enumerate}
	\item 1--2月:调研模拟框架的原理,学习基础知识;
	\item 3月:实现简单的几何构建,实现粒子源、物理列表和材料构建;
	\item 4月:完成几何构建,实现简单的数字化模块;
	\item 5月:实现关键探测器的数字化功能,分析输出结果。
\end{enumerate}
% }}}

\chapter{CEE实验概况}
% {{{ CEE实验概况

\section{物理目标}

\section{主要探测器}

\section{实验布局}

% }}}

\chapter{模拟框架的主体结构}
% {{{ 模拟框架的主体结构

\section{几何构建}

\section{材料模块}

\section{数字化模块}

\section{电磁场模块}

\section{粒子源与物理列表}
% }}}

\chapter{关键技术难点}
% {{{ 关键技术难点

\section{关键探测器的数字化实现}

\subsection{MRPC}
\subsection{TPC}
\subsection{MWDC}

\section{径迹重建的基本原理}

% }}}

\chapter{模拟样例与分析}

% {{{ 模拟样例与分析
% }}}

\chapter*{结论}

% {{{ 结论
% }}}

% 致谢
% !Mode:: "TeX:UTF-8"
\chapter*{致谢}
\addcontentsline{toc}{chapter}{致谢}
本文是在清华大学物理系的实验核物理研究组肖志刚教授的指导下完成的。
其中,径迹重建的方法源于课题组博士生吕黎明的实验分析课题;
几何构建和数字化等工作得到了同为研究组本科生的秦智同学的帮助。

我在此感谢所有为我本次工作提供知识上和其他方面帮助的人士。
\cleardoublepage

% 参考文献
\include{data/reference}

% 附录
\appendix
%\include{data/appendix1-faq}
%\include{data/appendix2-contactus}
\end{document}
